\section{Vorgehen}

Um sich auf die Thematik vorzubereiten hat sich jedes Gruppenmitglied mit dem grundlegenden Domänenwissen in Kryptowährungen angegeignet. Weitere Details betreffend dem Erarbeiten von Domänenwissen wird in diesem Bericht nicht weiter erwähnt.

Als Datengrundlage werden die Preisverlaufsdaten von den Kryptowährungen Bitcoin, Ethereum, Basic Attention Token, Litecoin und der Indexe Gold \parencite{yf2024gld}, S\&P500 \parencite{yf2024sp500} mit der yfinance Python-Library von Yahoo Finance heruntergeladen. Für die Kryptowährungen sind Daten ab 2014 von  Yahoo-Finance vorhanden und für Gold und S\&P500 gibt es ältere Daten, doch diese werden nicht verwendet. Zu bemerken ist, dass Kryptowährungen auch am Wochenende gehandelt werden, die Indexe Gold und S\&P500 sind an herkömmlichen Börsen und werden also an Wochenenden und Feiertagen nicht gehandelt.

In einem ersten Schritt müssen die Zeitreihen zuerst richtig verstanden werden, dehalb wird eine gründliche explorative Datenanalyse durchgeführt. Anschliessend werden die Daten mit einem geeigneten Preprocessing für die weitere Verwendung vorbereitet.
Zum Preprocessing gehört auch das splitten der Daten in Trainings-, Validierungs- und Testdatensätze. Wichtig zu betonen ist, dass diese Splits chronologisch verlaufen sind. Dies ist wichtig bei Zeitreihen. Folgende Splits werden gemacht:

\begin{itemize}
    \item Die ersten 1800 Tage als Trainingsdatensatz
    \item Tag 1801 bis 3100 (1300 Tage) als Validierungsdatensatz
    \item Tag 3101 bis 3462 (362 Tage) als Testdatensatz
\end{itemize}

Anhand stochastischen Prozessen werden dann die Zeitreihen modelliert. Hierbei wird der Fokus auf Bitcoin und Ethereum gesetzt. Die Zeitreihen werden einzeln dekomponiert und die Residuen mit einem ARIMA-Model vorhergesagt. Ein weiterer Approach der Zeitreihenanalysen kann mit Deep Learning verfolgt werden. Es werden dort Siamese Networks, LSTMs mit Attention und Temporal Convolutional Networks (TCN) behandelt.

Um zu prüfen, wie gut die implementierten Deep Learning-Modelle performen, wird ein Trader implementiert, der je nach Ausgabe der Modelle einkauft, haltet oder verkauft. Bei der Validierung und beim Testen wird die Walk-Forward-Methode angewendet. Dabei stehen dem Modell alle Daten bis zum aktuellen Tag ($T$) zur Verfügung, um den Preis des nächsten Tages ($T+1$) vorherzusagen. \parencite{Gray2018}
